\documentclass[11pt]{article} % use this if you are printing one side


\usepackage[cm]{fullpage}
\usepackage{bytefield}
\usepackage{float}

\addtolength{\parskip}{-1mm}
\setlength{\floatsep}{3pt plus 1.0pt minus 2.0pt}
\setlength{\textfloatsep}{3pt plus 1.0pt minus 2.0pt}
\setlength{\intextsep}{4pt plus 1.0pt minus 2.0pt}

\newcommand{\myname}{Wei Shi}
\newcommand{\myemail}{weishi@stanford.edu}

\newcommand{\packetHeader}[1]{
        \bitheader{0,7-8,15-16,23-24,31} \\
        \bitbox{8}{#1} \bitbox{8}{0x00} 
        \bitbox{8}{Version} \bitbox{8}{Node Type}\\
        \bitbox{32}{Node ID} \\
        \bitbox{32}{Sequence Number} \\
        \bitbox{32}{File ID} 
}

\author{\myname \\ \texttt{\myemail} }


\begin{document}
\title{CS244B Project 2 - Replicated Filesystem}
\date{\today}
\maketitle

\section{Protocol specification}

\subsection{Packet Header}
\begin{figure}[htbp]
\centering
    \begin{bytefield}{32}
        \packetHeader{Op-code}
    \end{bytefield}
	\caption{Packet Header}
\end{figure}
\begin{table} [H]
\centering
\begin{tabular}{l p{0.7\textwidth}}
		{\bf Op-code} & A byte uniquely identifying the packet type\\
		{\bf Version} & Version number reserved for future protocol. 
        It remains 0 in this version. \\
		{\bf Node Type} & 0 for client; 1 for server.\\ 
		{\bf Node ID} & ID of the packet sender.
        Randomly generated and unique among all nodes.\\
		{\bf Sequence number} & A counter of the packet. 
        Wrap around to 0 upon overflow.\\
		{\bf File ID} & Unique ID identifying the file under operation.
	\end{tabular}
\end{table}

\subsection{OpenFile}
\begin{figure}[htbp]
\centering
	\begin{bytefield}{32}
        \packetHeader{0x00}\\
        \bitbox{32}{Filename[0-3]}\\
		\bitbox{32}{...}\\
		\bitbox{32}{Filename[124-127]}
	\end{bytefield}
	\caption{OpenFile Packet}
\end{figure}
\begin{table} [H]
\centering
	\begin{tabular}{l p{0.7\textwidth}}
		{\bf Filename} & Name of the file to open. 
        128-byte string including null terminator.\\
	\end{tabular}
\end{table}
The client sends an OpenFile packet when it wants the server to open that
file and prepare for writing.

\subsection{OpenFileAck}
\begin{figure}[htbp]
\centering
	\begin{bytefield}{32}
        \packetHeader{0x01}\\
        \bitbox{8}{Status}
	\end{bytefield}
	\caption{OpenFileAck Packet}
\end{figure}
\begin{table} [H]
\centering
	\begin{tabular}{l p{0.7\textwidth}}
		{\bf Status} & 0 if the file is successfully opened. Negative number on failure.\\ 
	\end{tabular}
\end{table}
When the server receives OpenFile packet, it tries to open that file.
If it successes, it returns the OpenFileAck packet with status 0.
If it fails for any reason, it returns the OpenFileAck packet with negative status.

\subsection{WriteBlock}
\begin{figure}[htbp]
\centering
	\begin{bytefield}{32}
        \packetHeader{0x02}\\
		\bitbox{32}{Block ID}\\
		\bitbox{32}{Offset}\\
		\bitbox{32}{Size}\\
        \bitbox{32}{Payload(Variable size)}
	\end{bytefield}
	\caption{WriteBlock Packet}
\end{figure}
\begin{table} [H]
\centering
	\begin{tabular}{l p{0.7\textwidth}}
		{\bf Block ID} & Unique ID of the WriteBlock request.\\ 
		{\bf Offset} & Offset in the file to write to.\\ 
		{\bf Size} & Size of the block to write. 512 Max.\\ 
		{\bf Payload} & Block content to write.\\ 
	\end{tabular}
\end{table}
The client sends the WriteBlock packet when a new write is issued or when it receives
a ResendBlock packet from server before commit. 
It does not wait for the server to respond.

\subsection{CommitPrepare}
\begin{figure}[H]
\centering
	\begin{bytefield}{32}
        \packetHeader{0x03}\\
		\bitbox{32}{Num Blocks($n<=128$)}\\
        \bitbox{32}{Block ID[0]}\\
        \bitbox{32}{Block ID...}\\
        \bitbox{32}{Block ID[n-1]}
	\end{bytefield}
	\caption{CommitPrepare Packet}
\end{figure}
\begin{table} [H]
\centering
	\begin{tabular}{l p{0.7\textwidth}}
		{\bf Num Blocks} & Number of blocks to commit.\\ 
		{\bf Block ID} & Unique ID of the WriteBlock request.\\ 
	\end{tabular}
\end{table}
The client sends CommitPrepare packet when a commit is issued. This includes a list
of blocks to commit. When the server receives the list, it will check this against
its local block list. This is to ensure all servers have consistent blocks to commit.

\subsection{ResendBlock}
\begin{figure}[htbp]
\centering
	\begin{bytefield}{32}
        \packetHeader{0x04}\\
		\bitbox{32}{Num Blocks($n<=128$)}\\
        \bitbox{32}{Missing Block ID[0]}\\
        \bitbox{32}{Missing Block ID...}\\
        \bitbox{32}{Missing Block ID[n-1]}
	\end{bytefield}
	\caption{ResendBlock Packet}
\end{figure}
\begin{table} [H]
\centering
	\begin{tabular}{l p{0.7\textwidth}}
		{\bf Num Blocks} & Number of blocks to resend.\\ 
		{\bf Missing Block ID} & Unique ID of the WriteBlock request.\\ 
	\end{tabular}
\end{table}
When the server finds it is missing some of the blocks to commit from CommitPrepare
packet, it will send the ResendBlock with the missing block IDs to the client
for retransmission.

\subsection{CommitReady} 
\begin{figure}[htbp]
\centering
	\begin{bytefield}{32}
        \packetHeader{0x05}
	\end{bytefield}
	\caption{CommitReady Packet}
\end{figure}
The server sends CommitReady when it receives a CommitPrepare and all local 
outstanding WriteBlock requests match the list in CommitPrepare.

\subsection{Commit}
\begin{figure}[H]
\centering
	\begin{bytefield}{32}
        \packetHeader{0x06}
	\end{bytefield}
	\caption{Commit Packet}
\end{figure}
When client receives CommitReady packet from all servers, 
it sends Commit packet to instruct all servers to perform the commit.

\subsection{CommitSuccess}
\begin{figure}[H]
\centering
	\begin{bytefield}{32}
        \packetHeader{0x07}
	\end{bytefield}
	\caption{CommitSuccess Packet}
\end{figure}
When the server receives Commit packet, 
it commits local changes and replies with a CommitSuccess packet.

\subsection{Abort}
\begin{figure}[H]
\centering
	\begin{bytefield}{32}
        \packetHeader{0x08}
	\end{bytefield}
	\caption{Abort Packet}
\end{figure}
The client sends abort packet to all servers to discard the uncommited changes.

\subsection{Close}
\begin{figure}[H]
\centering
	\begin{bytefield}{32}
        \packetHeader{0x09}
	\end{bytefield}
	\caption{Close Packet}
\end{figure}
The client sends close packet to all servers to close the file.

\section{Protocol Details}

\subsection{Node ID generation}
Each node randomly generates a 32-bit integer as its ID.
By using a good seed for the random number generator, 
nodes can bring down the chance of ID conflicts significantly. 
The probability of two nodes generating the same id becomes negligible 
(in a 32-bit address space, the chance of collision is 5.4210109e-20).
Since in a more realistic setting, those servers will probably be manually configured 
to have a unique ID. So no ID conflict resolving is done in the protocol.

\subsection{File ID generation}
File ID is a unique identifier of the file under operation. 
It must be different for the two files that are opened back-to-back.
It's up to the client implementation to assign a File ID for every OpenFile call.
In general, this can be incremented at each new call or randomly generated.
File ID reusing is discouraged in order to prevent confusion caused by packet delay.

\subsection{Client/Server inconsistency}
There are multiple calls that do not require Server responses, 
which introduces inconsistency during packet loss.
\begin{itemize}
    \item WriteBlock(): Since WriteBlock request can be lost, 
        server might have fewer blocks.
        We use the block list in CommitPrepare packet to ensure consistency.
    \item Abort(): Since Abort request can be lost, 
        server might have extra blocks.
        We use the block list in CommitPrepare packet to ensure consistency.
\end{itemize}

\subsection{Operation flow}
\begin{itemize}
    \item On OpenFile(), client sends OpenFile and waits for OpenFileAck from all servers.
    \item Server creates file and sends OpenFileAck(success). 
        Otherwise it sends OpenFileAck(failure), if server cannot open that file.
    \item On WriteBlock(), client sends WriteBlock and proceed without response from servers.
    \item On Commit(), client sends CommitPrepare and waits for CommitReady or ResendBlock.
        If all servers respond with CommitReady, client sends Commit.
        If some server asks for blocks with ResendBlock, client sends WriteBlock
        and restarts the CommitPrepare operation.
    \item Server sends CommitReady when it has all the blocks in CommitPrepare.
        Otherwise, it sends ResendBlock to client asking for those missing blocks.
    \item Server commits all blocks when it receives Commit, 
        and responds with CommitSuccess.
    \item Client returns 0 to commit() call when it receives CommitSuccess from all servers.\
    \item On Abort(), client sends Abort to all server and discards uncommited blocks.
    \item Server discards all uncommited blocks when it receives Abort.
\end{itemize}

\subsection{Timing}
\subsubsection{Open File}
When the client sends OpenFile packet, it waits for the OpenFileAck from servers.
It fails immediately, if one OpenFileAck has a failure status. 
It fails after 1 second, if it does not get responses from all servers.

\subsubsection{CommitPrepare}
When the client sends CommitPrepare packet, it waits for the CommitReady 
or ResendBlock from servers.
This operation fails after 1 second, if it does not get responses from all servers.
This will cause the commit() call to fail.

\subsubsection{Commit}
When the client sends Commit packet, it waits for the CommitSuccess from servers.
This operation fails after 1 second, if it does not get responses from all servers.
This will cause the commit() call to fail.


\end{document}
